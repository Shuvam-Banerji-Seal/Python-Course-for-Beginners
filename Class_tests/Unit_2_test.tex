
\documentclass[9pt]{article}
\usepackage{geometry}
\geometry{a4paper, margin=0.75in}
\usepackage{multicol}
\usepackage{amsmath}
\usepackage{amssymb}
\usepackage{graphicx}
\usepackage{listings}
\usepackage{xcolor}
\usepackage{fancyhdr}
\usepackage{helvet}
\usepackage{courier}

% Set different font for the exam
\renewcommand{\familydefault}{\sfdefault}

% Code styling for Python
\definecolor{codegreen}{rgb}{0,0.6,0}
\definecolor{codegray}{rgb}{0.5,0.5,0.5}
\definecolor{codepurple}{rgb}{0.58,0,0.82}
\definecolor{backcolour}{rgb}{0.95,0.95,0.92}

\lstdefinestyle{mystyle}{
    backgroundcolor=\color{backcolour},
    commentstyle=\color{codegreen},
    keywordstyle=\color{magenta},
    numberstyle=\tiny\color{codegray},
    stringstyle=\color{codepurple},
    basicstyle=\ttfamily\footnotesize,
    breakatwhitespace=false,
    breaklines=true,
    captionpos=b,
    keepspaces=true,
    numbers=left,
    numbersep=5pt,
    showspaces=false,
    showstringspaces=false,
    showtabs=false,
    tabsize=2
}

\lstset{style=mystyle}

\pagestyle{fancy}
\fancyhf{}
\fancyhead[L]{\textbf{Computational Thinking and Programming - I}}
\fancyhead[R]{\textbf{Mock Test - I}}
\fancyfoot[C]{\thepage}

\author{Shuvam Banerji Seal}
\title{Computational Thinking and Programming - I \\ Mock Test - I}
\date{\today}

\begin{document}

\maketitle

\begin{center}
\hrule
\vspace{1em}
\textbf{Instructions for the Examinee}
\vspace{1em}
\hrule
\end{center}

\begin{enumerate}
\item \textbf{Total Questions:} The paper consists of 150 multiple-choice questions (MCQs).
\item \textbf{Total Time:} You have a total of 120 minutes (2 hours) to complete the examination. This means you have approximately 48 seconds per question. Time management is key!
\item \textbf{Marking Scheme:}
\begin{itemize}
\item For each correct answer, you will be awarded \textbf{+4 marks}.
\item For each incorrect answer, \textbf{1 mark will be deducted} from your total score.
\item No marks will be awarded or deducted for unattempted questions.
\end{itemize}
\item \textbf{Question Format:} Each question has four options (A, B, C, D). Only \textbf{one} of these options is correct.
\item \textbf{Syllabus Coverage:} The questions are based on the "Computational Thinking and Programming - I" syllabus, testing your theoretical knowledge, logical reasoning, and coding skills in Python.
\item \textbf{Difficulty Level:} The questions are designed to be of intermediate difficulty, with some tricky ones to challenge you. Read each question and its options carefully before answering.
\item \textbf{A Touch of Humor:} Don't be surprised if you find a few questions that are a bit funny. They are designed to be a small break from the intensity. However, they still require a correct answer!
\item \textbf{Rough Work:} You can use the blank pages provided for rough work.
\item \textbf{Answer Sheet:} Mark your answers on the separate OMR sheet provided. Ensure your marking is clear and within the designated circle.
\end{enumerate}

\begin{center}
\textbf{All the best!}
\end{center}

\newpage

\begin{multicols}{2}

\section*{Questions}

\begin{enumerate}
% Introduction to Problem-solving
\item A programmer is creating a flowchart for a program that checks if a number is even or odd. Which of the following symbols would be most appropriate to represent the decision "is the number divisible by 2"?
\begin{enumerate}
\item[A)] Rectangle (Process)
\item[B)] Parallelogram (Input/Output)
\item[C)] Diamond (Decision)
\item[D)] Oval (Start/End)
\end{enumerate}

\item Consider the following pseudocode to find the largest of three numbers (A, B, C):
\begin{verbatim}
1. START
2. READ A, B, C
3. IF A > B THEN
4.   IF A > C THEN
5.     PRINT A
6.   ELSE
7.     PRINT C
8.   ENDIF
9. ELSE
10.  IF B > C THEN
11.    PRINT B
12.  ENDIF
13. ENDIF
14. STOP
\end{verbatim}
For which set of inputs will this pseudocode fail to produce the correct output?
\begin{enumerate}
    \item[A)] A=10, B=20, C=30
    \item[B)] A=30, B=20, C=10
    \item[C)] A=10, B=30, C=20
    \item[D)] A=20, B=10, C=30
\end{enumerate}

\item Breaking down a complex problem into smaller, more manageable sub-problems is a key step in problem-solving. This process is known as:
\begin{enumerate}
    \item[A)] Decomposition
    \item[B)] Abstraction
    \item[C)] Debugging
    \item[D)] Compilation
\end{enumerate}

% Familiarization with the basics of Python programming
\item Why was the Python programming language named "Python"?
\begin{enumerate}
    \item[A)] After the snake, because it's a powerful and flexible language.
    \item[B)] It's an acronym for "Programming Your Thoroughly Helpful Online Notes".
    \item[C)] After the British comedy troupe "Monty Python's Flying Circus".
    \item[D)] It was randomly generated by a computer program.
\end{enumerate}

\item Which of the following is an invalid identifier in Python?
\begin{enumerate}
    \item[A)] \_my\_variable
    \item[B)] myVariable
    \item[C)] 2nd\_variable
    \item[D)] my\_variable\_2
\end{enumerate}

\item What is the difference between interactive mode and script mode in Python?
\begin{enumerate}
    \item[A)] Interactive mode is for writing long programs, while script mode is for single commands.
    \item[B)] Interactive mode executes commands immediately, while script mode saves commands in a file for later execution.
    \item[C)] Interactive mode does not show errors, while script mode does.
    \item[D)] There is no difference; the terms are used interchangeably.
\end{enumerate}

\item In Python, a variable \texttt{x = 10} is defined. The \texttt{10} in this context is referred to as:
\begin{enumerate}
    \item[A)] An l-value
    \item[B)] An r-value
    \item[C)] A functor
    \item[D)] A keyword
\end{enumerate}

% Knowledge of data types
\item Which of the following data types in Python is immutable?
\begin{enumerate}
    \item[A)] List
    \item[B)] Dictionary
    \item[C)] Set
    \item[D)] Tuple
\end{enumerate}

\item What will be the output of the following code?
\begin{lstlisting}[language=Python]
x = [1, 2, [3, 4]]
y = x.copy()
y[2][0] = 5
print(x)
\end{lstlisting}
\begin{enumerate}
\item[A)] [1, 2, [3, 4]]
\item[B)] [1, 2, [5, 4]]
\item[C)] [1, 2, 5]
\item[D)] A TypeError will be raised.
\end{enumerate}

\item The data type that can store a value of either \texttt{True} or \texttt{False} is:
\begin{enumerate}
    \item[A)] \texttt{integer}
    \item[B)] \texttt{float}
    \item[C)] \texttt{complex}
    \item[D)] \texttt{boolean}
\end{enumerate}

\item Consider \texttt{a = 10} and \texttt{b = 10.0}. What will be the output of \texttt{a == b} and \texttt{a is b} respectively?
\begin{enumerate}
    \item[A)] \texttt{True}, \texttt{True}
    \item[B)] \texttt{True}, \texttt{False}
    \item[C)] \texttt{False}, \texttt{True}
    \item[D)] \texttt{False}, \texttt{False}
\end{enumerate}

% Operators
\item What is the value of the expression \texttt{10 // 3 ** 2 * 2}?
\begin{enumerate}
    \item[A)] \texttt{2}
    \item[B)] \texttt{1}
    \item[C)] \texttt{2.22}
    \item[D)] \texttt{0}
\end{enumerate}

\item What will be the output of the following Python code?
\begin{lstlisting}[language=Python]
x = 5
x += 3 * 2
print(x)
\end{lstlisting}
\begin{enumerate}
\item[A)] 16
\item[B)] 11
\item[C)] 13
\item[D)] 30
\end{enumerate}

\item Which of the following expressions evaluates to \texttt{True}?
\begin{enumerate}
    \item[A)] \texttt{not (5 > 3 and 2  3) and (2  3 or 2  1)}
\end{enumerate}

\item What is the output of \texttt{'p' in 'python'} and \texttt{'P' in 'python'}?
\begin{enumerate}
    \item[A)] \texttt{True}, \texttt{True}
    \item[B)] \texttt{True}, \texttt{False}
    \item[C)] \texttt{False}, \texttt{True}
    \item[D)] \texttt{False}, \texttt{False}
\end{enumerate}

% Expressions, statement, type conversion, and input/output
\item Which of the following is an example of explicit type conversion?
\begin{enumerate}
    \item[A)] \texttt{5 + 2.5}
    \item[B)] \texttt{int('10')}
    \item[C)] \texttt{'Hello ' + 'World'}
    \item[D)] \texttt{x = 10}
\end{enumerate}

\item What will be the output of \texttt{print(int(10.7) + float(10))}?
\begin{enumerate}
    \item[A)] \texttt{20.7}
    \item[B)] \texttt{20.0}
    \item[C)] \texttt{21}
    \item[D)] A \texttt{ValueError}
\end{enumerate}

\item A user enters \texttt{42} when the following code is run. What will be the output?
\begin{lstlisting}[language=Python]
age = input("Enter your age: ")
print(age * 2)
\end{lstlisting}
\begin{enumerate}
\item[A)] 84
\item[B)] 4242
\item[C)] 42 42
\item[D)] A TypeError
\end{enumerate}

% Errors
\item An error that occurs when the program is syntactically correct but does something other than what the programmer intended is called a:
\begin{enumerate}
    \item[A)] Syntax Error
    \item[B)] Runtime Error
    \item[C)] Logical Error
    \item[D)] Compilation Error
\end{enumerate}

\item What type of error will be raised by the following code?
\begin{lstlisting}[language=Python]
x = 10 / 0
\end{lstlisting}
\begin{enumerate}
\item[A)] SyntaxError
\item[B)] ZeroDivisionError
\item[C)] TypeError
\item[D)] NameError
\end{enumerate}

% Flow of Control - Conditional statements
\item What will be the output of the following code snippet?
\begin{lstlisting}[language=Python]
x = 0
if x > 0:
    print("Positive")
elif x == 0:
    print("Zero")
else:
    print("Negative")
\end{lstlisting}
\begin{enumerate}
\item[A)] Positive
\item[B)] Zero
\item[C)] Negative
\item[D)] No output
\end{enumerate}

\item To sort three numbers \texttt{a, b, c} in ascending order, which of the following conditional structures would be the most straightforward to implement?
\begin{enumerate}
    \item[A)] A single \texttt{if} statement
    \item[B)] An \texttt{if-else} statement
    \item[C)] A series of \texttt{if} statements
    \item[D)] A nested \texttt{if-elif-else} structure
\end{enumerate}

% Flow of Control - Iterative Statement
\item What is the output of the following code?
\begin{lstlisting}[language=Python]
for i in range(1, 6, 2):
    print(i, end=' ')
\end{lstlisting}
\begin{enumerate}
\item[A)] 1 2 3 4 5
\item[B)] 1 3 5
\item[C)] 1 2 3 4 5 6
\item[D)] 2 4 6
\end{enumerate}

\item How many times will the "inner loop" be executed in the following code?
\begin{lstlisting}[language=Python]
for i in range(3):
    for j in range(2):
        # inner loop
        print(i, j)
\end{lstlisting}
\begin{enumerate}
\item[A)] 5
\item[B)] 6
\item[C)] 3
\item[D)] 2
\end{enumerate}

\item What will the following \texttt{while} loop print?
\begin{lstlisting}[language=Python]
i = 5
while i > 0:
    i -= 1
    if i == 2:
        continue
    print(i, end=' ')
\end{lstlisting}
\begin{enumerate}
\item[A)] 4 3 1 0
\item[B)] 4 3 2 1 0
\item[C)] 4 3 1
\item[D)] 5 4 3 1 0
\end{enumerate}

\item If a programmer wants to write a loop that executes forever (until the program is interrupted), which of the following is the most common way to do it?
\begin{enumerate}
    \item[A)] \texttt{for i in range(infinity):}
    \item[B)] \texttt{while True:}
    \item[C)] \texttt{while 1 pi}
    \item[D)] \texttt{Pi()}
\end{enumerate}

\item If you use the statement \texttt{from math import sqrt}, how would you calculate the square root of 16?
\begin{enumerate}
    \item[A)] \texttt{math.sqrt(16)}
    \item[B)] \texttt{sqrt(16)}
    \item[C)] \texttt{math->sqrt(16)}
    \item[D)] \texttt{sqrt.math(16)}
\end{enumerate}

\item What is the primary advantage of using \texttt{import random as rd}?
\begin{enumerate}
    \item[A)] It makes the program run faster.
    \item[B)] It imports all functions from the \texttt{random} module.
    \item[C)] It provides a shorter alias, which is useful for long module names.
    \item[D)] It is the only way to import a module.
\end{enumerate}

\item A potential issue with using \texttt{from module import *} is that:
\begin{enumerate}
    \item[A)] It is much slower than a standard import.
    \item[B)] It can lead to name clashes if functions from the module have the same name as variables in your code.
    \item[C)] It does not import variables from the module, only functions.
    \item[D)] It is deprecated and will be removed in future Python versions.
\end{enumerate}

\item The trigonometric functions in the \texttt{math} module, such as \texttt{math.sin()} and \texttt{math.cos()}, expect their input to be in which unit?
\begin{enumerate}
    \item[A)] Degrees
    \item[B)] Radians
    \item[C)] Gradians
    \item[D)] Steradians
\end{enumerate}

\item What is the output of \texttt{print(math.ceil(5.2), math.floor(5.8))} after importing the \texttt{math} module?
\begin{enumerate}
    \item[A)] \texttt{5 6}
    \item[B)] \texttt{6 5}
    \item[C)] \texttt{5.0 5.0}
    \item[D)] \texttt{6.0 5.0}
\end{enumerate}

\item What are the return types of \texttt{3 ** 2} and \texttt{math.pow(3, 2)} respectively?
\begin{enumerate}
    \item[A)] \texttt{int}, \texttt{int}
    \item[B)] \texttt{float}, \texttt{int}
    \item[C)] \texttt{float}, \texttt{float}
    \item[D)] \texttt{int}, \texttt{float}
\end{enumerate}

\item What will be the result of \texttt{math.sqrt(-9)}?
\begin{enumerate}
    \item[A)] \texttt{-3.0}
    \item[B)] \texttt{3.0}
    \item[C)] A \texttt{ValueError}
    \item[D)] A \texttt{TypeError}
\end{enumerate}

\item A programmer is feeling lazy and wants to calculate the absolute value of \texttt{-20}. Which of these functions will give the result \texttt{20.0}?
\begin{enumerate}
    \item[A)] \texttt{abs(-20)}
    \item[B)] \texttt{math.abs(-20)}
    \item[C)] \texttt{math.fabs(-20)}
    \item[D)] \texttt{fabs(-20)}
\end{enumerate}

\item A student's code has a logical error. They decide to fix it by putting \texttt{import random} at the top and running the code again, hoping for a different outcome. This debugging technique is fondly known as:
\begin{enumerate}
    \item[A)] Stochastic Programming
    \item[B)] Hope-Driven Development
    \item[C)] Nondeterministic Analysis
    \item[D)] A terrible idea that almost never works
\end{enumerate}

\item Which function from the \texttt{random} module would you use to get a random integer between 1 and 10, inclusive?
\begin{enumerate}
    \item[A)] \texttt{random.random(1, 10)}
    \item[B)] \texttt{random.randrange(1, 10)}
    \item[C)] \texttt{random.randint(1, 10)}
    \item[D)] \texttt{random.choice(1, 10)}
\end{enumerate}

\item Which of the following code snippets can produce the integer \texttt{5} as output?
\begin{enumerate}
    \item[A)] \texttt{random.randrange(5)}
    \item[B)] \texttt{random.randint(1, 4)}
    \item[C)] \texttt{random.randrange(1, 5)}
    \item[D)] \texttt{random.randint(5, 10)}
\end{enumerate}

\item What is the key difference between \texttt{random.randrange(a, b)} and \texttt{random.randint(a, b)}?
\begin{enumerate}
    \item[A)] \texttt{randrange} includes \texttt{b} in the range, while \texttt{randint} does not.
    \item[B)] \texttt{randint} includes \texttt{b} in the range, while \texttt{randrange} does not.
    \item[C)] \texttt{randrange} can only be used with a step value.
    \item[D)] \texttt{randint} returns a float, while \texttt{randrange} returns an integer.
\end{enumerate}

\item To simulate a fair coin flip that results in either "Heads" or "Tails", which is the most suitable approach?
\begin{enumerate}
    \item[A)] \texttt{random.randint(0, 1)}
    \item[B)] \texttt{random.choice(["Heads", "Tails"])}
    \item[C)] \texttt{random.random()}
    \item[D)] \texttt{random.randrange(2)}
\end{enumerate}

\item What does \texttt{random.random()} return?
\begin{enumerate}
    \item[A)] A random integer.
    \item[B)] A random floating point number in the range [0.0, 1.0).
    \item[C)] A random floating point number in the range [0.0, 1.0].
    \item[D)] A random byte.
\end{enumerate}

\item Given the list \texttt{data = [10, 20, 30, 40]}, what is the result of \texttt{statistics.mean(data)}?
\begin{enumerate}
    \item[A)] \texttt{25.0}
    \item[B)] \texttt{25}
    \item[C)] \texttt{30}
    \item[D)] \texttt{100}
\end{enumerate}

\item What is the median of the list \texttt{nums = [5, 2, 8, 1, 9]}?
\begin{enumerate}
    \item[A)] \texttt{8}
    \item[B)] \texttt{5}
    \item[C)] \texttt{2}
    \item[D)] \texttt{5.2}
\end{enumerate}

\item What will \texttt{statistics.mode([1, 2, 2, 3, 3, 4])} return?
\begin{enumerate}
    \item[A)] \texttt{2}
    \item[B)] \texttt{3}
    \item[C)] \texttt{[2, 3]}
    \item[D)] A \texttt{StatisticsError}
\end{enumerate}

\item For the dataset \texttt{[1, 1, 2, 5, 6]}, which of the following is true?
\begin{enumerate}
    \item[A)] mean > median > mode
    \item[B)] median > mean > mode
    \item[C)] mode > mean > median
    \item[D)] mean > median = mode
\end{enumerate}

\item What is the output of the following code?
\begin{lstlisting}[language=Python]
import math
data = [1, 4, 9, 16]
result = [int(math.sqrt(x)) for x in data if x > 5]
print(sum(result))
\end{lstlisting}
\begin{enumerate}
\item[A)] 7
\item[B)] 7.0
\item[C)] 9
\item[D)] 10
\end{enumerate}

\item What is a possible output of this code snippet?
\begin{lstlisting}[language=Python]
import random
chars = 'xyz'
print(random.choice(chars) + str(random.randint(1,1)))
\end{lstlisting}
\begin{enumerate}
\item[A)] y2
\item[B)] z1
\item[C)] x2
\item[D)] z
\end{enumerate}

\item What is the result of the following expression?
\begin{lstlisting}[language=Python]
int(math.log10(100)) + math.floor(3.14)
\end{lstlisting}
\begin{enumerate}
\item[A)] 5
\item[B)] 5.0
\item[C)] 5.14
\item[D)] 6
\end{enumerate}

\item You have a list of numbers \texttt{L}. Which of the following is guaranteed to raise an error if \texttt{L} is empty?
\begin{enumerate}
    \item[A)] \texttt{len(L)}
    \item[B)] \texttt{sum(L)}
    \item[C)] \texttt{sorted(L)}
    \item[D)] \texttt{statistics.mean(L)}
\end{enumerate}

\item A programmer wants to get a random even number between 2 and 10 (inclusive). Which of the following is the best way to do this?
\begin{enumerate}
    \item[A)] \texttt{random.randint(2, 10)}
    \item[B)] \texttt{random.randrange(2, 10, 2)}
    \item[C)] \texttt{random.randrange(2, 11, 2)}
    \item[D)] \texttt{random.choice([2, 4, 6, 8, 10])}
\end{enumerate}

\item What is the output of the code below?
\begin{lstlisting}[language=Python]
import math
print(math.factorial(4) / math.pow(2, 3))
\end{lstlisting}
\begin{enumerate}
\item[A)] 3
\item[B)] 3.0
\item[C)] 4
\item[D)] 4.0
\end{enumerate}

% Mixed/Tricky Questions
\item What is the final value of x?
\begin{lstlisting}[language=Python]
x = 100
for i in range(x):
    if i % 10 == 0:
        continue
    if i == 50:
        break
    x -= 1
\end{lstlisting}
\begin{enumerate}
\item[A)] 50
\item[B)] 55
\item[C)] 45
\item[D)] 60
\end{enumerate}

\item A programmer explains their code to a rubber duck sitting on their desk until they spot the error in their logic. This sacred and time-honored tradition is called:
\begin{enumerate}
    \item[A)] Quack-Oriented Programming
    \item[B)] Rubber Duck Debugging
    \item[C)] Fowl Play Analysis
    \item[D)] The Duck-typing Method
\end{enumerate}

\item Predict the output of the following code snippet.
\begin{lstlisting}[language=Python]
data = {(1, 2): 'a', 'key': 'b'}
data[1, 2] = 'c'
data['key'] = (3, 4)
print(data[1, 2], data['key'][0])
\end{lstlisting}
\begin{enumerate}
\item[A)] a 3
\item[B)] c 3
\item[C)] c 4
\item[D)] A TypeError is raised.
\end{enumerate}

\item What is the output of this code?
\begin{lstlisting}[language=Python]
s = "  pytHon_is_fun  "
result = s.strip().capitalize().replace('_', ' ')
print(result)
\end{lstlisting}
\begin{enumerate}
\item[A)] Python is fun
\item[B)] PytHon is fun
\item[C)] Python Is Fun
\item[D)] Python is fun
\end{enumerate}

\item Which of the following expressions will evaluate to \texttt{False}?
\begin{lstlisting}[language=Python]
a = 257
b = 257
x = 5
y = 5
\end{lstlisting}
\begin{enumerate}
\item[A)] x is y
\item[B)] a is not b
\item[C)] a == b
\item[D)] a is b
\end{enumerate}

\item What is the final value of the list \texttt{L}?
\begin{lstlisting}[language=Python]
L = [1, 2, 3, 4, 5]
for x in L:
    if x % 2 == 0:
        L.remove(x)
\end{lstlisting}
\begin{enumerate}
\item[A)] [1, 3, 5]
\item[B)] [1, 3, 4, 5]
\item[C)] [1, 2, 3, 5]
\item[D)] [1, 3, 4]
\end{enumerate}

\item Predict the output:
\begin{lstlisting}[language=Python]
d = {'b': 1, 'a': 2, 'c': 0}
s_items = sorted(d.items())
print(s_items[1][0])
\end{lstlisting}
\begin{enumerate}
\item[A)] a
\item[B)] b
\item[C)] 1
\item[D)] 2
\end{enumerate}

\item If a programmer accidentally tries to calculate \texttt{math.factorial(-1)}, their computer will most likely:
\begin{enumerate}
    \item[A)] Return \texttt{0}.
    \item[B)] Enter a state of existential crisis.
    \item[C)] Return \texttt{-1}.
    \item[D)] Raise a \texttt{ValueError}.
\end{enumerate}

\item What is the value of \texttt{result}?
\begin{lstlisting}[language=Python]
result = 0
i = 1
while i <= 10:
    result += i
    i += 2
print(result)
\end{lstlisting}
\begin{enumerate}
\item[A)] 55
\item[B)] 30
\item[C)] 25
\item[D)] 36
\end{enumerate}

\item Consider the flowchart symbol for a process (a rectangle). In the context of writing a Python program, this symbol most closely corresponds to:
\begin{enumerate}
    \item[A)] An \texttt{if} statement
    \item[B)] An \texttt{input()} function call
    \item[C)] A comment
    \item[D)] An assignment statement like \texttt{x = y + 1}
\end{enumerate}

\item What will the following code print?
\begin{lstlisting}[language=Python]
my_tuple = (10, 20, [30, 40])
my_tuple[2][0] = 35
print(my_tuple)
\end{lstlisting}
\begin{enumerate}
\item[A)] (10, 20, [35, 40])
\item[B)] (10, 20, [30, 40])
\item[C)] It will raise a TypeError because tuples are immutable.
\item[D)] It will raise an IndexError.
\end{enumerate}

\item In the expression \texttt{x = y}, \texttt{x} is the l-value and \texttt{y} is the r-value. Which of the following is true?
\begin{enumerate}
    \item[A)] The l-value must be a memory location, while the r-value must be a value.
    \item[B)] The r-value must be a memory location, while the l-value must be a value.
    \item[C)] Both l-value and r-value must be memory locations.
    \item[D)] The terms are interchangeable and have no real meaning.
\end{enumerate}

\item What is the output of the code?
\begin{lstlisting}[language=Python]
print(bool(None), bool([]), bool('False'))
\end{lstlisting}
\begin{enumerate}
\item[A)] False False False
\item[B)] False True True
\item[C)] False False True
\item[D)] True True True
\end{enumerate}

\item The code \texttt{val = 1\_000\_000} is executed. What is the type and value of \texttt{val}?
\begin{enumerate}
    \item[A)] \texttt{str}, \texttt{'1\_000\_000'}
    \item[B)] \texttt{int}, \texttt{1000000}
    \item[C)] It's a \texttt{SyntaxError}.
    \item[D)] \texttt{tuple}, \texttt{(1, 0, 0, 0, 0, 0, 0)}
\end{enumerate}

\item What will be printed on the screen?
\begin{lstlisting}[language=Python]
for i in range(2):
    print(i, end=' ')
else:
    print("Loop finished!")
\end{lstlisting}
\begin{enumerate}
\item[A)] 0 1
\item[B)] 0 1 Loop finished!
\item[C)] 0 1 2 Loop finished!
\item[D)] Loop finished!
\end{enumerate}

\item Now, what will this code print?
\begin{lstlisting}[language=Python]
for i in range(5):
    print(i, end=' ')
    if i == 2:
        break
else:
    print("Loop finished!")
\end{lstlisting}
\begin{enumerate}
\item[A)] 0 1 2 Loop finished!
\item[B)] 0 1 2 3 4 Loop finished!
\item[C)] 0 1 2
\item[D)] 0 1
\end{enumerate}

\item What is the result of \texttt{type('abc'.partition('d'))}?
\begin{enumerate}
    \item[A)] \texttt{str}
    \item[B)] \texttt{list}
    \item[C)] \texttt{tuple}
    \item[D)] \texttt{dict}
\end{enumerate}

\item Which method call on a string \texttt{s} is guaranteed to return a list?
\begin{enumerate}
    \item[A)] \texttt{s.find('a')}
    \item[B)] \texttt{s.strip()}
    \item[C)] \texttt{s.join(['a','b'])}
    \item[D)] \texttt{s.split()}
\end{enumerate}

\item What is the value of \texttt{z}?
\begin{lstlisting}[language=Python]
x = [1, 2]
y = [3, 4]
z = [x, y]
x[0] = 5
\end{lstlisting}
\begin{enumerate}
\item[A)] [[1, 2], [3, 4]]
\item[B)] [[5, 2], [3, 4]]
\item[C)] [5, 2, 3, 4]
\item[D)] [1, 2, 3, 4]
\end{enumerate}

\item You are given a list of employees: \texttt{employees = ['Alice', 'Bob', 'Charlie']}. Which of the following will add 'David' to the list so that the list becomes \texttt{['Alice', 'Bob', 'David', 'Charlie']}?
\begin{enumerate}
    \item[A)] \texttt{employees.insert(2, 'David')}
    \item[B)] \texttt{employees.insert(3, 'David')}
    \item[C)] \texttt{employees.append('David')}
    \item[D)] \texttt{employees[2] = 'David'}
\end{enumerate}

\item What is the output of the following code?
\begin{lstlisting}[language=Python]
d = {}
d[1] = 1
d['1'] = 2
d[1.0] = 3
print(d['1'], d[1])
\end{lstlisting}
\begin{enumerate}
\item[A)] 2 1
\item[B)] 2 3
\item[C)] 1 2
\item[D)] A KeyError is raised.
\end{enumerate}

\item Which of the following methods removes a key from a dictionary and returns its value, but raises an error if the key is not found?
\begin{enumerate}
    \item[A)] \texttt{get()}
    \item[B)] \texttt{popitem()}
    \item[C)] \texttt{pop()}
    \item[D)] \texttt{clear()}
\end{enumerate}

\item What will \texttt{statistics.mode(['cat', 'dog', 'dog', 'cat', 'tiger'])} return?
\begin{enumerate}
    \item[A)] \texttt{'cat'}
    \item[B)] \texttt{'dog'}
    \item[C)] \texttt{['cat', 'dog']}
    \item[D)] A \texttt{StatisticsError} is raised.
\end{enumerate}

\item What is the value of \texttt{result}?
\begin{lstlisting}[language=Python]
import random
random.seed(42) # Guarantees the same sequence of "random" numbers
L = [10, 20, 30, 40, 50]
result = random.choice(L[1:4])
\end{lstlisting}
\begin{enumerate}
\item[A)] 10
\item[B)] 40
\item[C)] 30
\item[D)] 20
\end{enumerate}

\item The expression \texttt{True or False and not True} evaluates to:
\begin{enumerate}
    \item[A)] \texttt{True}
    \item[B)] \texttt{False}
    \item[C)] \texttt{None}
    \item[D)] A \texttt{SyntaxError}
\end{enumerate}

\item What is the most likely reason a programmer would use the \texttt{pass} keyword?
\begin{enumerate}
    \item[A)] To stop the execution of a program.
    \item[B)] To skip the current iteration of a loop.
    \item[C)] As a placeholder for code they intend to write later, to avoid syntax errors.
    \item[D)] To make their code pass all tests. (If only it were that easy!)
\end{enumerate}

\item What is the output of this code snippet?
\begin{lstlisting}[language=Python]
msg = "hello"
msg.upper()
msg = msg + " world"
print(msg)
\end{lstlisting}
\begin{enumerate}
\item[A)] HELLO world
\item[B)] Hello world
\item[C)] hello world
\item[D)] HELLO WORLD
\end{enumerate}

\item What will be the value of \texttt{mean}?
\begin{lstlisting}[language=Python]
import statistics
data = (1, 2, 3, 4, '5')
try:
    mean = statistics.mean(data)
except TypeError:
    mean = "Error"
\end{lstlisting}
\begin{enumerate}
\item[A)] 3
\item[B)] 3.0
\item[C)] "Error"
\item[D)] The program crashes with a TypeError.
\end{enumerate}

\item What's the output?
\begin{lstlisting}[language=Python]
t = (1, 2, 3) * 2
L = [1, 2, 3] * 2
print(t[3] == L[3])
\end{lstlisting}
\begin{enumerate}
\item[A)] True
\item[B)] False
\item[C)] TypeError
\item[D)] IndexError
\end{enumerate}

\item Which of the following cannot be a dictionary key?
\begin{enumerate}
    \item[A)] A string
    \item[B)] A tuple
    \item[C)] An integer
    \item[D)] A list
\end{enumerate}

% Final Set - More Tricky/Comprehensive Questions
\item What is the value of the expression 5 * 1**2 + 3 // 2?
\begin{enumerate}
\item[A)] 7
\item[B)] 6
\item[C)] 13
\item[D)] 8.5
\end{enumerate}

\item Predict the output:
\begin{lstlisting}[language=Python]
s = 'abcdef'
print(s[5:1:-2])
\end{lstlisting}
\begin{enumerate}
\item[A)] 'fd'
\item[B)] 'fdb'
\item[C)] 'fd'
\item[D)] 'eca'
\end{enumerate}

\item What is the output of the following code?
\begin{lstlisting}[language=Python]
x = 10
y = 20
x, y = y, x
x = x - 10
y = y + 5
print(x, y)
\end{lstlisting}
\begin{enumerate}
\item[A)] 10 25
\item[B)] 20 15
\item[C)] 10 15
\item[D)] 20 25
\end{enumerate}

\item Which line of code will raise a runtime error?
\begin{lstlisting}[language=Python]
# Line 1
my_list = [1, 2, 3]
# Line 2
my_tuple = (1, 2, 3)
# Line 3
print(my_list[3])
# Line 4
print(my_tuple[2])
\end{lstlisting}
\begin{enumerate}
\item[A)] Line 1
\item[B)] Line 2
\item[C)] Line 3
\item[D)] Line 4
\end{enumerate}

\item A programmer writes an infinite loop by mistake. The most common phrase they will utter upon discovering this is:
\begin{enumerate}
    \item[A)] "Eureka! Perpetual motion!"
    \item[B)] "It's not a bug, it's a feature."
    \item[C)] "Oh, that's why my fan is so loud."
    \item[D)] "Exactly as planned."
\end{enumerate}

\item What is the final content of \texttt{my\_dict}?
\begin{lstlisting}[language=Python]
keys = ['a', 'b', 'c']
values = [1, 2]
my_dict = dict(zip(keys, values))
my_dict['c'] = 3
\end{lstlisting}
\begin{enumerate}
\item[A)] \{'a': 1, 'b': 2\}
\item[B)] \{'a': 1, 'b': 2, 'c': None\}
\item[C)] \{'a': 1, 'b': 2, 'c': 3\}
\item[D)] A ValueError is raised.
\end{enumerate}

\item What will be printed?
\begin{lstlisting}[language=Python]
i = 0
while i < 5:
    print(i, end=" ")
    i += 1
    if i == 3:
        break
else:
    print("done")
\end{lstlisting}
\begin{enumerate}
\item[A)] 0 1 2 done
\item[B)] 0 1 2 3 done
\item[C)] 0 1 2
\item[D)] 0 1 2 3 4 done
\end{enumerate}

\item What is the output of \texttt{(None == 0, None is None, None == None)}?
\begin{enumerate}
    \item[A)] \texttt{(True, True, True)}
    \item[B)] \texttt{(False, True, True)}
    \item[C)] \texttt{(False, False, True)}
    \item[D)] \texttt{(True, False, False)}
\end{enumerate}

\item What is the output of the following code?
\begin{lstlisting}[language=Python]
t = ([1], [2])
L = t[0]
L.append(3)
print(t)
\end{lstlisting}
\begin{enumerate}
\item[A)] ([1], [2])
\item[B)] ([1, 3], [2])
\item[C)] ([1], [2], 3)
\item[D)] A TypeError is raised.
\end{enumerate}

\item What is the result of \texttt{len("".join([]))}?
\begin{enumerate}
    \item[A)] \texttt{0}
    \item[B)] \texttt{1}
    \item[C)] \texttt{2}
    \item[D)] A \texttt{TypeError} is raised.
\end{enumerate}

\item What is printed by the following code?
\begin{lstlisting}[language=Python]
def mystery(s):
    return s.isalnum() or s.isspace()

print(mystery("Hello World"), mystery("HelloWorld1"), mystery("   "))
\end{lstlisting}
\begin{enumerate}
\item[A)] False True True
\item[B)] True True True
\item[C)] False False True
\item[D)] True False True
\end{enumerate}

\item What is the output?
\begin{lstlisting}[language=Python]
import math
print(int(math.pow(2, 3.5)))
\end{lstlisting}
\begin{enumerate}
\item[A)] 11
\item[B)] 11.31
\item[C)] 8
\item[D)] A ValueError is raised.
\end{enumerate}

\item A developer wants to check if a string \texttt{s} starts with 'http' or 'https'. Which is the most efficient way?
\begin{enumerate}
    \item[A)] \texttt{s.startswith('http') or s.startswith('https')}
    \item[B)] \texttt{s.startswith(('http', 'https'))}
    \item[C)] \texttt{s[0:4] == 'http'}
    \item[D)] \texttt{s.find('http') == 0}
\end{enumerate}

\item What is the value of \texttt{count} after this loop?
\begin{lstlisting}[language=Python]
count = 0
for i in range(5):
    for j in range(i):
        count += 1
\end{lstlisting}
\begin{enumerate}
\item[A)] 25
\item[B)] 20
\item[C)] 15
\item[D)] 10
\end{enumerate}

\item Which of these statements is syntactically incorrect in Python?
\begin{enumerate}
    \item[A)] \texttt{x, y, z = 1, 2, 3}
    \item[B)] \texttt{x = 1, 2, 3}
    \item[C)] \texttt{x, y = 1, 2, 3}
    \item[D)] \texttt{x = y = z = 1}
\end{enumerate}

\item What is the median of the data \texttt{[10, 40, 20, 50]}?
\begin{enumerate}
    \item[A)] \texttt{20}
    \item[B)] \texttt{40}
    \item[C)] \texttt{30}
    \item[D)] \texttt{35}
\end{enumerate}

\item What is the output of the following snippet?
\begin{lstlisting}[language=Python]
a = [10, 20]
b = [10, 20]
c = a
print(a is b, a == b, a is c)
\end{lstlisting}
\begin{enumerate}
\item[A)] True True True
\item[B)] False True False
\item[C)] False True True
\item[D)] True False True
\end{enumerate}

\item When a programmer says their code is "elegant", it usually means:
\begin{enumerate}
    \item[A)] It's written in cursive.
    \item[B)] It uses very rare and obscure libraries.
    \item[C)] It solves a complex problem in a simple, clear, and efficient way.
    \item[D)] It has no comments, so its brilliance is self-evident.
\end{enumerate}

\item What is the value of \texttt{L[0][1]} after this code executes?
\begin{lstlisting}[language=Python]
L1 = [1, 2, 3]
L2 = L1[:]
L = [L1, L2]
L1[1] = 4
\end{lstlisting}
\begin{enumerate}
\item[A)] 2
\item[B)] 4
\item[C)] 3
\item[D)] 1
\end{enumerate}

\item What does \texttt{random.randrange(10, 0, -2)} generate?
\begin{enumerate}
    \item[A)] An even number from 2 to 10 inclusive.
    \item[B)] An odd number from 1 to 9 inclusive.
    \item[C)] An even number from 0 to 8 inclusive.
    \item[D)] A \texttt{ValueError}.
\end{enumerate}

\item What is the output of the code?
\begin{lstlisting}[language=Python]
print('xyz'.find('y', 1, 2))
\end{lstlisting}
\begin{enumerate}
\item[A)] 1
\item[B)] -1
\item[C)] True
\item[D)] IndexError
\end{enumerate}

\item You want to create a dictionary from a list of strings, where each string is a key and its length is the value. Which is the most "Pythonic" way?
\begin{enumerate}
    \item[A)] \texttt{d = \{\}; for s in L: d[s] = len(s)}
    \item[B)] \texttt{d = \{s: len(s) for s in L\}}
    \item[C)] \texttt{d = dict(zip(L, map(len, L)))}
    \item[D)] All of the above are valid and reasonably Pythonic.
\end{enumerate}

\item Which of the following is true about \texttt{del}?
\begin{enumerate}
    \item[A)] It is a function that returns the deleted item.
    \item[B)] It can only be used on lists.
    \item[C)] It is a statement that can remove items from a list/dictionary or unbind a variable.
    \item[D)] \texttt{del x[0]} is the same as \texttt{x.pop(0)}.
\end{enumerate}

\item What is the result of \texttt{max("apple", "banana", key=len)}?
\begin{enumerate}
    \item[A)] \texttt{'apple'}
    \item[B)] \texttt{'banana'}
    \item[C)] \texttt{6}
    \item[D)] \texttt{KeyError}
\end{enumerate}

\item What is the output?
\begin{lstlisting}[language=Python]
d = {'a': 1, 'b': 2}
d.update(b=3, c=4)
print(d)
\end{lstlisting}
\begin{enumerate}
\item[A)] \{'a': 1, 'b': 2, 'c': 4\}
\item[B)] \{'a': 1, 'b': 3, 'c': 4\}
\item[C)] A SyntaxError
\item[D)] \{'b': 3, 'c': 4\}
\end{enumerate}

\item What is the most significant difference between pseudocode and a flowchart?
\begin{enumerate}
    \item[A)] Pseudocode is for simple logic, flowcharts are for complex logic.
    \item[B)] Pseudocode is textual, while a flowchart is a graphical representation of logic.
    \item[C)] Only flowcharts can show loops.
    \item[D)] Pseudocode is language-specific.
\end{enumerate}

\item What is the output?
\begin{lstlisting}[language=Python]
print(1 > 2 == False)
\end{lstlisting}
\begin{enumerate}
\item[A)] True
\item[B)] False
\item[C)] 1
\item[D)] A SyntaxError because of operator chaining.
\end{enumerate}

\item Which of the following concepts is NOT a fundamental part of the "Introduction to Problem-solving" unit?
\begin{enumerate}
    \item[A)] Developing an Algorithm
    \item[B)] Testing and Debugging
    \item[C)] Choosing a cloud provider
    \item[D)] Analyzing the problem
\end{enumerate}

\item What is the output of this final question?
\begin{lstlisting}[language=Python]
import math
import random
# Don't worry, the exam is almost over
the_end_is_near = True
if the_end_is_near:
    # You did great!
    result = math.ceil(random.random())
else:
    # This won't happen
    result = math.floor(random.random())

print(result)
\end{lstlisting}
\begin{enumerate}
\item[A)] 0
\item[B)] 1
\item[C)] 0.0
\item[D)] 1.0
\end{enumerate}

\item The primary fuel for a programmer is:
\begin{enumerate}
    \item[A)] Logic and algorithms.
    \item[B)] Electricity.
    \item[C)] Unwavering self-confidence.
    \item[D)] Coffee. (And sometimes pizza).
\end{enumerate}

"""\item What is the output of `print(0.1 + 0.2 == 0.3)`?
\begin{enumerate}
\item[A)] True
\item[B)] False
\item[C)] Error
\item[D)] Depends on the Python version
\end{enumerate}

\item What is the value of `x` after this code executes?
\begin{lstlisting}[language=Python]
x = [1, 2, 3]
x.append(x.pop(1))
\end{lstlisting}
\begin{enumerate}
\item[A)] `[1, 2, 3]`
\item[B)] `[1, 3, 2]`
\item[C)] `[2, 1, 3]`
\item[D)] `[1, 3]`
\end{enumerate}

\item What is the output of the following?
\begin{lstlisting}[language=Python]
s = "hello"
print(s.find('l', s.find('l') + 1))
\end{lstlisting}
\begin{enumerate}
\item[A)] 2
\item[B)] 3
\item[C)] -1
\item[D)] Error
\end{enumerate}

\item What is the result of `bool("False")`?
\begin{enumerate}
\item[A)] True
\item[B)] False
\item[C)] Error
\item[D)] None
\end{enumerate}

\item What is the output of the following code?
\begin{lstlisting}[language=Python]
d = {'a': 1, 'b': 2}
d.setdefault('c', 3)
d.setdefault('a', 4)
print(d)
\end{lstlisting}
\begin{enumerate}
\item[A)] `{'a': 1, 'b': 2, 'c': 3}`
\item[B)] `{'a': 4, 'b': 2, 'c': 3}`
\item[C)] `{'a': 1, 'b': 2, 'c': 3, 'a': 4}`
\item[D)] Error
\end{enumerate}

\item What is the output of `print(2 * 3 ** 2)`?
\begin{enumerate}
\item[A)] 18
\item[B)] 36
\item[C)] 12
\item[D)] 64
\end{enumerate}

\item What is the output of the following code?
\begin{lstlisting}[language=Python]
t = (1, 2, 3)
t[1] = 4
print(t)
\end{lstlisting}
\begin{enumerate}
\item[A)] `(1, 4, 3)`
\item[B)] `(1, 2, 3, 4)`
\item[C)] Error
\item[D)] `(4, 2, 3)`
\end{enumerate}

\item What is the output of `print("hello".replace("l", "L", 1))`?
\begin{enumerate}
\item[A)] `heLLo`
\item[B)] `heLlo`
\item[C)] `Hello`
\item[D)] `heLLo`
\end{enumerate}

\item What is the output of the following code?
\begin{lstlisting}[language=Python]
a = [1, 2, 3]
b = a
b.append(4)
print(a)
\end{lstlisting}
\begin{enumerate}
\item[A)] `[1, 2, 3]`
\item[B)] `[1, 2, 3, 4]`
\item[C)] `[1, 2, 3, [4]]`
\item[D)] Error
\end{enumerate}

\item What is the output of `print(isinstance(1, float))`?
\begin{enumerate}
\item[A)] True
\item[B)] False
\item[C)] Error
\item[D)] `None`
\end{enumerate}

\item What is the output of the following code?
\begin{lstlisting}[language=Python]
s = "a,b,c"
print(s.split(',', 1))
\end{lstlisting}
\begin{enumerate}
\item[A)] `['a', 'b,c']`
\item[B)] `['a', 'b', 'c']`
\item[C)] `['a,b', 'c']`
\item[D)] `['a', 'b c']`
\end{enumerate}

\item What is the output of `print(1 == True)`?
\begin{enumerate}
\item[A)] True
\item[B)] False
\item[C)] Error
\item[D)] `None`
\end{enumerate}

\item What is the output of the following code?
\begin{lstlisting}[language=Python]
d = {'a': 1}
print(d.get('b', 0))
\end{lstlisting}
\begin{enumerate}
\item[A)] 1
\item[B)] 0
\item[C)] Error
\item[D)] `None`
\end{enumerate}

\item What is the output of `print(3 * 'abc' + 'xyz')`?
\begin{enumerate}
\item[A)] `abcabcabcxyz`
\item[B)] `abc xyz`
\item[C)] `abc abc abc xyz`
\item[D)] Error
\end{enumerate}

\item What is the output of the following code?
\begin{lstlisting}[language=Python]
a = {1, 2, 3}
b = {3, 4, 5}
print(a.union(b))`
\end{lstlisting}
\begin{enumerate}
\item[A)] `{1, 2, 3, 4, 5}`
\item[B)] `{1, 2, 3, 3, 4, 5}`
\item[C)] `{3}`
\item[D)] `{1, 2, 4, 5}`
\end{enumerate}

\item What is the output of `print(int("10", 2))`?
\begin{enumerate}
\item[A)] 10
\item[B)] 2
\item[C)] 4
\item[D)] 8
\end{enumerate}

\item What is the output of the following code?
\begin{lstlisting}[language=Python]
a = [1, 2, 3]
b = a[:]
print(a is b)
\end{lstlisting}
\begin{enumerate}
\item[A)] True
\item[B)] False
\item[C)] Error
\item[D)] `None`
\end{enumerate}

\item What is the output of `print("Hello, World!".partition(','))`?
\begin{enumerate}
\item[A)] `('Hello', ',', ' World!')`
\item[B)] `['Hello', ',', ' World!']`
\item[C)] `('Hello', ' World!')`
\item[D)] `('Hello,', ' ', 'World!')`
\end{enumerate}

\item What is the output of the following code?
\begin{lstlisting}[language=Python]
a = 1
def my_func():
    a = 2
    print(a)
my_func()
print(a)
\end{lstlisting}
\begin{enumerate}
\item[A)] 2 2
\item[B)] 2 1
\item[C)] 1 2
\item[D)] 1 1
\end{enumerate}

\item What is the output of `print(float('inf') > 1e308)`?
\begin{enumerate}
\item[A)] True
\item[B)] False
\item[C)] Error
\item[D)] `None`
\end{enumerate}

\item What is the output of the following code?
\begin{lstlisting}[language=Python]
a = [1, 2, 3]
del a[1:]
print(a)
\end{lstlisting}
\begin{enumerate}
\item[A)] `[1]`
\item[B)] `[2, 3]`
\item[C)] `[1, 2]`
\item[D)] `[1, 3]`
\end{enumerate}

\item What is the output of `print('abc'.rjust(5, '*'))`?
\begin{enumerate}
\item[A)] `**abc`
\item[B)] `abc**`
\item[C)] `*abc*`
\item[D)] `abc`
\end{enumerate}

\item What is the output of the following code?
\begin{lstlisting}[language=Python]
a = {1, 2}
b = {1, 2, 3}
print(a.issubset(b))
\end{lstlisting}
\begin{enumerate}
\item[A)] True
\item[B)] False
\item[C)] Error
\item[D)] `None`
\end{enumerate}

\item What is the output of `print(complex(2, 3) * complex(3, 2))`?
\begin{enumerate}
\item[A)] `(0, 13)`
\item[B)] `(12, 13j)`
\item[C)] `(0 + 13j)`
\item[D)] `13j`
\end{enumerate}

\item What is the output of the following code?
\begin{lstlisting}[language=Python]
a = [1, 2, 3]
b = [4, 5, 6]
print(a + b)
\end{lstlisting}
\begin{enumerate}
\item[A)] `[1, 2, 3, 4, 5, 6]`
\item[B)] `[5, 7, 9]`
\item[C)] `[[1, 2, 3], [4, 5, 6]]`
\item[D)] Error
\end{enumerate}

\item What is the output of `print(round(2.5))`?
\begin{enumerate}
\item[A)] 2
\item[B)] 3
\item[C)] 2.0
\item[D)] 3.0
\end{enumerate}

\item What is the output of the following code?
\begin{lstlisting}[language=Python]
a = "hello"
print(a.zfill(8))
\end{lstlisting}
\begin{enumerate}
\item[A)] `000hello`
\item[B)] `hello000`
\item[C)] `00hello0`
\item[D)] `hello`
\end{enumerate}

\item What is the output of `print(1 != True)`?
\begin{enumerate}
\item[A)] True
\item[B)] False
\item[C)] Error
\item[D)] `None`
\end{enumerate}

\item What is the output of the following code?
\begin{lstlisting}[language=Python]
a = [1, 2, 3]
a.clear()
print(a)
\end{lstlisting}
\begin{enumerate}
\item[A)] `[]`
\item[B)] `None`
\item[C)] `[0, 0, 0]`
\item[D)] Error
\end{enumerate}

\item What is the output of `print(0 or 1)`?
\begin{enumerate}
\item[A)] 0
\item[B)] 1
\item[C)] True
\item[D)] False
\end{enumerate}

\item What is the output of the following code?
\begin{lstlisting}[language=Python]
a = [1, 2, 3]
b = a.copy()
b[0] = 4
print(a)
\end{lstlisting}
\begin{enumerate}
\item[A)] `[4, 2, 3]`
\item[B)] `[1, 2, 3]`
\item[C)] `[1, 2, 3, 4]`
\item[D)] Error
\end{enumerate}

\item What is the output of `print(2 and 3)`?
\begin{enumerate}
\item[A)] 2
\item[B)] 3
\item[C)] True
\item[D)] False
\end{enumerate}

\item What is the output of the following code?
\begin{lstlisting}[language=Python]
a = "hello"
print(a.endswith("o"))
\end{lstlisting}
\begin{enumerate}
\item[A)] True
\item[B)] False
\item[C)] Error
\item[D)] `None`
\end{enumerate}

\item What is the output of `print(1 < 2 < 3)`?
\begin{enumerate}
\item[A)] True
\item[B)] False
\item[C)] Error
\item[D)] `None`
\end{enumerate}

\item What is the output of the following code?
\begin{lstlisting}[language=Python]
a = [1, 2, 3]
print(a.pop())
\end{lstlisting}
\begin{enumerate}
\item[A)] 1
\item[B)] 2
\item[C)] 3
\item[D)] `[1, 2]`
\end{enumerate}

\item What is the output of `print(not 0)`?
\begin{enumerate}
\item[A)] True
\item[B)] False
\item[C)] 1
\item[D)] -1
\end{enumerate}

\item What is the output of the following code?
\begin{lstlisting}[language=Python]
a = "hello"
print(a.find("z"))
\end{lstlisting}
\begin{enumerate}
\item[A)] 0
\item[B)] -1
\item[C)] Error
\item[D)] `None`
\end{enumerate}

\item What is the output of `print(1 is 1.0)`?
\begin{enumerate}
\item[A)] True
\item[B)] False
\item[C)] Error
\item[D)] `None`
\end{enumerate}

\item What is the output of the following code?
\begin{lstlisting}[language=Python]
a = [1, 2, 3]
a.reverse()
print(a)
\end{lstlisting}
\begin{enumerate}
\item[A)] `[3, 2, 1]`
\item[B)] `[1, 2, 3]`
\item[C)] `None`
\item[D)] Error
\end{enumerate}

\item What is the output of `print(1 and 0)`?
\begin{enumerate}
\item[A)] 1
\item[B)] 0
\item[C)] True
\item[D)] False
\end{enumerate}

\end{enumerate}

\end{multicols}


ewpage

% Answer Sheet
\pagestyle{plain}
\begin{center}
{\huge 	extbf{ANSWER SHEET}} 
\vspace{0.5em}
{\Large Computational Thinking and Programming - I \ Mock Test - I} 
\vspace{0.5em}
\hrule
\end{center}

\vspace{1em}

\begin{multicols}{5}
\small
\begin{enumerate}
\item C
\item A
\item A
\item C
\item C
\item B
\item B
\item D
\item B
\item D
\item B
\item D
\item B
\item A
\item B
\item B
\item B
\item B
\item C
\item B
\item B
\item D
\item B
\item B
\item C
\item A
\item B
\item C
\item C
\item A
\item A
\item B
\item A
\item D
\item D
\item C
\item C
\item B
\item C
\item C
\item A
\item C
\item B
\item C
\item C
\item C
\item D
\item B
\item C
\item C
\item B
\item C
\item C
\item B
\item B
\item A
\item C
\item D
\item B
\item A
\item A
\item A
\item B
\item C
\item B
\item B
\item B
\item C
\item B
\item B
\item B
\item D
\item C
\item C
\item D
\item B
\item C
\item B
\item D
\item C
\item D
\item B
\item A
\item B
\item D
\item C
\item A
\item B
\item B
\item C
\item B
\item B
\item B
\item C
\item D
\item D
\item B
\item B
\item C
\item A
\item D
\item C
\item A
\item B
\item A
\item C
\item B
\item B
\item D
\item B
\item C
\item C
\item A
\item A
\item C
\item C
\item D
\item C
\item B
\item A
\item C
\item C
\item C
\item B
\item C
\item C
\item D
\item B
\item A
\item A
\item B
\item B
\item A
\item B
\item C
\item C
\item D
\item C
\item C
\item B
\item B
\item B
\item C
\item C
\item B
\item B
\item C
\item B
\item B
\item C
\item B
\item D
\item B
\item B
\item B
\item A
\item A
\item A
\item C
\item B
\item B
\item B
\item A
\item A
\item B
\item A
\item A
\item B
\item C
\item A
\item A
\item C
\item A
\item A
\item B
\item A
\item B
\item B
\item A
\item A
\item C
\item A
\item B
\item B
\item A
\item B
\end{enumerate}
\end{multicols}""

\vspace{2em}

\begin{center}
\textbf{Marking Instructions:}
\begin{itemize}
\item Each correct answer: +4 marks
\item Each incorrect answer: -1 mark
\item No marks for unattempted questions
\item Maximum marks: 600
\item Minimum marks: -150
\end{itemize}
\end{center}

\end{document}


